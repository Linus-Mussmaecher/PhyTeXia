\documentclass[a4paper]{scrartcl}
\usepackage{phytexia}
\usepackage{floatrow}
\usepackage{booktabs}
\usepackage{hyperref}

\begin{document}

	\section*{Tests -- Symbols}

	\subsection*{Mana Symbols -- such as \mtgUR}
	The five colors of mana in Magic are \mtgwhite, \mtgblue, \mtgblack, \mtgred\ and \mtggreen .

	In costs, there is also generic mana, such as \mtggeneric{1}, \mtggeneric{20} or \mtggeneric{X}.
	There is no symbol for \mtggeneric{Y}.

	The cost of the card \emph{Supreme Verdict} is \mtggeneric{2}\mtgwhite\mtgblue .

	Each man cost also has Phyrexian equivalent, which can be paid with either 1 mana of the appropriate color or 2 life, e.g. \mtgUP{} can be paid with either \mtgU{} or 2 life.
	These symbols are \mtgWP, \mtgUP, \mtgBP, \mtgRP{} and \mtgGP.

	Similar to Phyrexian mana, so called \emph{2brid mana} can be paid with either 1 mana of a specific color or 2 mana of any color.
	The respective symbols for 2brid mana are \mtgTwoW, \mtgTwoU, \mtgTwoB, \mtgTwoR{} and \mtgTwoG.

	Lastly, one card in Magic (namely \emph{Ulalek, Fused Atrocity}) uses a special form of hybrid mana that can be paid with either mana of a specific color or colorless mana.
	Ulalek uses the symbols \mtgCW, \mtgCU, \mtgCB, \mtgCR{} and \mtgCG.

	Finally, we have the most basic form of hybrid mana that can be paid with one of two specific colors.
	These were introduced on the plane of Ravnica and are often associated with their guilds.
	They also each exist in a phyrexian variant.
	\begin{table}[h]
		\begin{tabular}{c c c}
			\toprule
			\textbf{Guild} & \textbf{Hybrid Mana Symbol} & \textbf{Phyrexian Hybrid Mana Symbol}\\
			\midrule
			Azorius  & \mtgWU & \mtgWUP \\
			Orzhov   & \mtgWB & \mtgWBP \\
			Dimir    & \mtgUB & \mtgUBP \\
			Izzet    & \mtgUR & \mtgURP \\
			Rakdos   & \mtgBR & \mtgBRP \\
			Golgari  & \mtgBG & \mtgBGP \\
			Boros    & \mtgRW & \mtgRWP \\
			Gruul    & \mtgRG & \mtgRGP \\
			Selesnya & \mtgGW & \mtgGWP \\
			Simic    & \mtgGU & \mtgGUP \\
			\bottomrule
		\end{tabular}
		\caption{The guilds of Ravnica and their associated hybrid mana symbols.}
	\end{table}

	Lastly, there is also snow mana (\mtgS), which is mana of any color that was produced by a snow source.
	\mtgSnow{} mostly appears in costs of abilities instead of cards.

	Notice that the size of symbols changes with the surrounding font size, as can be seen in the subsection title.

	\subsection*{Other Costs -- \mtgP{} and more}

	In addition to mana costs, players may also be required to tap or untap permanents as part of as a cost, as indicated by the \mtgTap{} and \mtgUntap{} symbols.
	\mtgT{} is a lot more common than \mtgQ.

	As an additional resource, \mtgEnergy{} was introduced in Kaladesh.
	Players put \mtgE{} counters on themselves as a resource that lasts between turns and may be generated and spent by different cards.
	The \mtgTicket{} symbol from Unfinity works similar, but appears only in Un-sets.

	As a play on Energy, the card Acornelia, Fashionable Filcher uses \mtgA-counters.
	\mtgAcorn works just like energy, but can only be created and spent by Acornelia.

	Lastly, Bloomburrow introduced the \mtgP{} symbol, which does not represent a resource that can be spent on different card.
	Instead, the five season cards that use them are each allocatd a budget of five \mtgPawprint{} that may be spent on three different modes of different costs.

	\subsection*{The Cost Command}

	Next, we use the \texttt{\textbackslash mtgcost}-Command to display the mana costs of some cards and abilities.

	\begin{table}[h]
		\begin{tabular}{l r}
			\toprule
			\textbf{Card}          & \textbf{Cost}                 \\
			\midrule
			Surpeme Verdict        & \mtgcost{{2}{U}W}             \\
			Rakshasa's Bargain     & \mtgcost{{2/B}{2/U}{2/G}}     \\
			Prismari Pledgemage    & \mtgcost{{U/R}{U/R}}          \\
			Gitaxian Probe         & \mtgcost{U/P}                 \\
			Dismemember            & \mtgcost{1B/PB/P}             \\
			Ulalek, Fused Atrocity & \mtgcost{C/W{C/U}C/B{C/R}C/G} \\
			Nahiri                 & \mtgcost{{1}{R}{R/W/P}{W}}    \\
			Sphinx's Revelation    & \mtgcost{XUU}                 \\
			Blinkmoth Infusion     & \mtgcost{12UU}                \\
			Gleemax                & \mtgcost{1000000}             \\
			Nightmare              & \mtgcost{XYZRR}               \\
			\bottomrule
       \end{tabular}
	\end{table}

	\begin{table}[h]
		\begin{tabular}{l r}
			\toprule
			\textbf{Ability}       & \textbf{Cost}   \\
			\midrule
			Acornelia              & \mtgcost{2{B},A}    \\
			Rage Extractor         & \mtgcost{H}     \\
			Season of Weaving      & \mtgcost{PP}     \\
			Mox Lotus Ability      & \mtgcost{100}   \\
			Glarb, Calamity Augur  & \mtgcost{T}     \\
			Knacksaw Clique        & \mtgcost{1U, Q} \\
			Whirler Virtuoso       & \mtgcost{EEE}   \\
			\bottomrule
       \end{tabular}
	\end{table}


	\subsection*{Type symbols}

	In the set Future Sight, Magic introduced symbols to represent its card types.
	These are:
	\begin{table}[h]
		\begin{tabular}{l c}
			\hline
			\textbf{Type} & \textbf{Symbol} \\
			\hline
			Creature & \mtgTypeCreature \\
			Sorcery & \mtgTypeSorcery \\
			Instant & \mtgTypeInstant \\
			Enchantment & \mtgTypeEnchantment \\
			Artifact & \mtgTypeArtifact \\
			Planeswalker & \mtgTypePlaneswalker \\
			Battle & \mtgTypeBattle \\
			Multi-Type Card & \mtgTypeMulti \\
			\hline
		\end{tabular}
	\end{table}

\end{document}


