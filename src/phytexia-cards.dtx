% \iffalse meta-comment
% This file contains the implementation for macros that display magic cards in \LaTeX.
% \fi
% \subsection{Displaying Cards}
% First, a basic card command that pull information about a card from the scryfall API and display its name.
% This will use fuzzy search and thus correct any misspellings.
%
%    \begin{macrocode}
\newcommand{\mtgcard}[1]{\directlua{PHYTEXIA.mtgcard("#1")}}
%    \end{macrocode}
%
% This next command will display the correctly formatted cost of the given card.
% 
%    \begin{macrocode}
\newcommand{\mtgcardcost}[1]{\directlua{PHYTEXIA.mtgcardcost("#1")}}
%    \end{macrocode}
%
% Finally, we can display the card as an image.
%    \begin{macrocode}
\newcommand{\mtgcardwidth}{4cm}
%    \end{macrocode}
%    \begin{macrocode}
\newcommand{\mtgsetcardwidth}[1]{\renewcommand{\mtgcardwidth}{#1}}
%    \end{macrocode}
% We define the command for displaying a card:
%    \begin{macrocode}
\newcommand{\mtgcardimg}[2][]{\directlua{PHYTEXIA.mtgcardimg("#2", "#1")}}
%    \end{macrocode}
% This command will form the basis of many other commands.
% The |gallery|-command displays multiple cards in centered rows.
%    \begin{macrocode}
\newcommand{\mtggallery}[1]{\textsc{#1}}
%    \end{macrocode}
